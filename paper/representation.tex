\section{Representations}
We make use of layers of representation between the surface text of a word problem and the learned model. 
The first of these is a domain-independent semantic representation derived directly from the text using pre-existing language processing tools.
This semantic representation is then transformed into a domain-specific intermediate representation of sets similar to the {\it Qset} representation used by \cite{konked2015}. 
Finally, a pair of intermediate representations are used to produce a vector for training. 
Each step of this process is described below.

\subsection{Semantic Representation}
\label{semantics}
The text of each sentence of a word problem is represented as a collection of predicates as provided by the Answer Constraint Engine parser (CITE woodley's website).
ACE parses text using HPSG grammars, and we use the English Resource Grammar (CITE Flickenger), a high precision, handbuilt grammar for the English language. 
The predications instantiate a collection of entity and event variables and express the different interactions of these variables according to the compositional meaning of the text. 
We track all predications in which a given entity variable appears as an argument.
We also track relations between entities appearing as arguments of the same predication.

Relevant to the task of solving math word problems are those entities that are related to numbers.
These entities often appear as the ARG1 of a CARD_REL whose CARG (and text span) is a number. 
If a CARD_REL has a non-entity ARG1, we associate it's CARG to an instantiated entity which shares the same label. 
These distinguished entities form our domain-specific intermediate representation, which for the sake of simplicity we will call {\it Number Entities}.

\subsection{Vector Representation}
\label{vectors}
As mentioned above, our system learns from triples consisting of two text quantities and an operator which appears in an equation leading to the correct solution. 
We represent a triple as a vector by considering the Number Entities corresponding to the text quantities in the triple. 
We evaluate several methods of vectorizing these semantic representations:
\begin{compactenum}
\item Predicates Only: For each of the ### predicates in the training data, the vector contains a 1 if the predicate is av
\item Predicate SVD
\item POS-based predicate word2vec distance
\end{compactenum}


